%% Overleaf			
%% Software Manual and Technical Document Template	
%% 									
%% This provides an example of a software manual created in Overleaf.

\documentclass{ol-softwaremanual}

% Packages used in this example
\usepackage{graphicx}  % for including images
\usepackage{microtype} % for typographical enhancements
% \usepackage{minted}    % for code listings
\usepackage{amsmath}   % for equations and mathematics
% \setminted{style=friendly,fontsize=\small}
% \renewcommand{\listoflistingscaption}{List of Code Listings}
\usepackage{hyperref}  % for hyperlinks
\usepackage[a4paper,top=4.2cm,bottom=4.2cm,left=3.5cm,right=3.5cm]{geometry} % for setting page size and margins

% Custom macros used in this example document
\newcommand{\doclink}[2]{\href{#1}{#2}\footnote{\url{#1}}}
\newcommand{\cs}[1]{\texttt{\textbackslash #1}}

% Frontmatter data; appears on title page
\title{DigitalCircularityToolkit}
\version{0.1}
\author{Keith JL}
% \softwarelogo{\includegraphics[width = 1.25\textwidth]{figures/front}}
\begin{document}

\maketitle


\tableofcontents
% \listoflistings
\newpage
\part{Introduction}
\section{Digital Circularity}

\section{DigitalCircularityToolkit}
\subsection{Minimum working example}

\part{Workflows}

\part{Component Reference}
\section{Objects}
\subsection{Object}
\begin{figure*}[h]
    \centering
    \includegraphics[width = .5\textwidth]{figures/Icons/OBJECT.pdf}
\end{figure*}

\subsection{LinearObject}
\begin{figure*}[h]
    \centering
    \includegraphics[width = .5\textwidth]{figures/Icons/LINEAROBJECT.pdf}
\end{figure*}

\subsection{PlanarObject}
\begin{figure*}[h]
    \centering
    \includegraphics[width = .5\textwidth]{figures/Icons/PLANAROBJECT.pdf}
\end{figure*}

\subsection{BoxObject}
\begin{figure*}[h]
    \centering
    \includegraphics[width = .5\textwidth]{figures/Icons/BOXOBJECT.pdf}
\end{figure*}

\subsection{SphericalObject}
\begin{figure*}[h]
    \centering
    \includegraphics[width = .5\textwidth]{figures/Icons/SPHERICALOBJECT.pdf}
\end{figure*}

\subsection{Utilities}

\subsubsection{ObjectProperties}
\begin{figure*}[h]
    \centering
    \includegraphics[width = .5\textwidth]{figures/Icons/OBJECTPROPERTIES.pdf}
\end{figure*}

\subsubsection{OverridePCA}
\begin{figure*}[h]
    \centering
    \includegraphics[width = .5\textwidth]{figures/Icons/OVERRIDEPCA.pdf}
\end{figure*}

\section{Sets}

\subsection{LinearSet}
\begin{figure*}[h]
    \centering
    \includegraphics[width = .5\textwidth]{figures/Icons/LINEARSET.pdf}
\end{figure*}

\subsection{PlanarSet}
\begin{figure*}[h]
    \centering
    \includegraphics[width = .5\textwidth]{figures/Icons/PLANARSET.pdf}
\end{figure*}

\subsection{BoxSet}
\begin{figure*}[h]
    \centering
    \includegraphics[width = .5\textwidth]{figures/Icons/BOXSET.pdf}
\end{figure*}

\subsection{SphereSet}
\begin{figure*}[h]
    \centering
    \includegraphics[width = .5\textwidth]{figures/Icons/SPHERESET.pdf}
\end{figure*}

\section{Characterization}
\subsection{FeatureVector}
\begin{figure*}[h]
    \centering
    \includegraphics[width = .5\textwidth]{figures/Icons/FEATUREVEC.pdf}
\end{figure*}

\subsection{LineScore}
\begin{figure*}[h]
    \centering
    \includegraphics[width = .5\textwidth]{figures/Icons/LINESCORE.pdf}
\end{figure*}

\subsection{PlaneScore}
\begin{figure*}[h]
    \centering
    \includegraphics[width = .5\textwidth]{figures/Icons/PLANESCORE.pdf}
\end{figure*}

\subsection{BoxScore}
\begin{figure*}[h]
    \centering
    \includegraphics[width = .5\textwidth]{figures/Icons/BOXSCORE.pdf}
\end{figure*}

\subsection{SphereScore}
\begin{figure*}[h]
    \centering
    \includegraphics[width = .5\textwidth]{figures/Icons/SPHERESCORE.pdf}
\end{figure*}

\subsection{RadialSignature}
\begin{figure*}[h]
    \centering
    \includegraphics[width = .5\textwidth]{figures/Icons/RADIALSIG.pdf}
\end{figure*}

\subsection{HarmonicAnalysisReal}
\begin{figure*}[h]
    \centering
    \includegraphics[width = .5\textwidth]{figures/Icons/HARMONICSREAL.pdf}
\end{figure*}

\subsection{HarmonicAnalysisComplex}
\begin{figure*}[h]
    \centering
    \includegraphics[width = .5\textwidth]{figures/Icons/HARMONICSCOMPLEX.pdf}
\end{figure*}

\section{Distance}
\subsection{EuclideanDistance}
\begin{figure*}[h]
    \centering
    \includegraphics[width = .5\textwidth]{figures/Icons/DISTANCESYMM.pdf}
\end{figure*}

\subsection{AsymmEuclideanDistance}
\begin{figure*}[h]
    \centering
    \includegraphics[width = .5\textwidth]{figures/Icons/DISTANCEASYMM.pdf}
\end{figure*}

\section{Matching}

\subsection{Hungarian}
\begin{figure*}[h]
    \centering
    \includegraphics[width = .5\textwidth]{figures/Icons/HUNGARIAN.pdf}
\end{figure*}

\subsection{Utilities}
\subsubsection{MatchLines}
\begin{figure*}[h]
    \centering
    \includegraphics[width = .5\textwidth]{figures/Icons/MATCHLINES.pdf}
\end{figure*}

\subsubsection{AlignToObject}
\begin{figure*}[h]
    \centering
    \includegraphics[width = .5\textwidth]{figures/Icons/ALIGNTOOBJECT.pdf}
\end{figure*}

\section{Utilities}

\subsection{Knoll}
\begin{figure*}[h]
    \centering
    \includegraphics[width = .5\textwidth]{figures/Icons/KNOLL.pdf}
\end{figure*}

\subsection{AlignToPlane}
\begin{figure*}[h]
    \centering
    \includegraphics[width = .5\textwidth]{figures/Icons/ALIGNTOPLANE.pdf}
\end{figure*}

\subsection{LineToVector}
\begin{figure*}[h]
    \centering
    \includegraphics[width = .5\textwidth]{figures/Icons/LINETOVECTOR.pdf}
\end{figure*}

\subsection{Normalize}
\begin{figure*}[h]
    \centering
    \includegraphics[width = .5\textwidth]{figures/Icons/NORMALIZE.pdf}
\end{figure*}

\subsection{PlanarHull}
\begin{figure*}[h]
    \centering
    \includegraphics[width = .5\textwidth]{figures/Icons/HULL.pdf}
\end{figure*}

\subsection{PlanarOutline}
\begin{figure*}[h]
    \centering
    \includegraphics[width = .5\textwidth]{figures/Icons/OUTLINE.pdf}
\end{figure*}

\subsection{RotatePCA}
\begin{figure*}[h]
    \centering
    \includegraphics[width = .5\textwidth]{figures/Icons/ROTATEPCA.pdf}
\end{figure*}

\subsection{ToPointCloud}
\begin{figure*}[h]
    \centering
    \includegraphics[width = .5\textwidth]{figures/Icons/TOPOINTLCOUD.pdf}
\end{figure*}

% \input{further-examples}

\end{document}
