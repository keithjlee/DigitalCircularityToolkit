%% Overleaf			
%% Software Manual and Technical Document Template	
%% 									
%% This provides an example of a software manual created in Overleaf.

\documentclass{ol-softwaremanual}

% Packages used in this example
\usepackage{graphicx}  % for including images
\usepackage{microtype} % for typographical enhancements
% \usepackage{minted}    % for code listings
\usepackage{amsmath}   % for equations and mathematics
% \setminted{style=friendly,fontsize=\small}
% \renewcommand{\listoflistingscaption}{List of Code Listings}
\usepackage{hyperref}  % for hyperlinks
\usepackage[a4paper,top=4.2cm,bottom=4.2cm,left=3.5cm,right=3.5cm]{geometry} % for setting page size and margins

% Custom macros used in this example document
\newcommand{\doclink}[2]{\href{#1}{#2}\footnote{\url{#1}}}
\newcommand{\cs}[1]{\texttt{\textbackslash #1}}

% Frontmatter data; appears on title page
\title{DigitalCircularityToolkit}
\version{0.0.1}
\author{Keith JL}
\softwarelogo{\includegraphics[width = 1.25\textwidth]{figures/front}}
\begin{document}

\maketitle


\tableofcontents
% \listoflistings
\newpage
\part{Introduction}
\section{Digital Circularity}

\section{DigitalCircularityToolkit}
\subsection{Minimum working example}

\part{Workflows}

\part{Component Reference}
\section{Objects}
\subsection{Object}

\subsection{LinearObject}

\subsection{PlanarObject}

\subsection{BoxObject}

\subsection{SphericalObject}

\subsection{Utilities}

\subsubsection{ObjectProperties}

\subsubsection{OverridePCA}

\section{Sets}

\subsection{LinearSet}

\subsection{PlanarSet}

\subsection{BoxSet}

\subsection{SphereSet}

\section{Characterization}
\subsection{FeatureVector}

\subsection{LineScore}

\subsection{PlaneScore}

\subsection{BoxScore}

\subsection{SphereScore}

\subsection{RadialSignature}

\subsection{HarmonicAnalysisReal}

\subsection{HarmonicAnalysisComplex}

\section{Distance}
\subsection{EuclideanDistance}

\subsection{AsymmEuclideanDistance}

\section{Matching}

\subsection{Hungarian}

\subsection{Utilities}
\subsubsection{MatchLines}

\subsubsection{AlignToObject}

\section{Utilities}

\subsection{Knoll}

\subsection{AlignToPlane}

\subsection{LineToVector}

\subsection{Normalize}

\subsection{PlanarHull}

\subsection{PlanarOutline}

\subsection{RotatePCA}

\subsection{ToPointCloud}

% \input{further-examples}

\end{document}
